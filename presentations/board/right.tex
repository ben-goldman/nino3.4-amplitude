\documentclass{beamer}

\title{The Impact of Anthropogenic Forcing on ENSO Amplitude}
\author{Ben Goldman}
\date{\today}

\usepackage{natbib}
\usepackage{tikz}
\usepackage{varwidth}
\usepackage{hyperref}
\usepackage[scale=1.4]{beamerposter}
\geometry{papersize={20in,48in}}
\usepackage{lipsum}

\usetheme{ben}

\newcommand{\myfig}[4]{
  \begin{figure}
    \centering
    \includegraphics[width=#3\textwidth]{figures/#1}
    \caption{#2}
    \label{fig:#4}
  \end{figure}
}

\renewcommand{\bibsection}{}

\begin{document}

\nocite{gistemp2019giss}
\nocite{lenssen2019improvements}
\nocite{ropelewski1987global}
\nocite{rayner2003global}
\nocite{lubbecke2014assessing}
\nocite{zhu2017reduced}
\nocite{zheng2017response}
\nocite{maher2018enso}
\nocite{kay2015community}
\nocite{danabasoglu2020community}

\begin{frame}

  \begin{block}{Wavelet Analysis (Current results)}
    \begin{itemize}
    \item Separate ENSO record into changes in period over time.
    \item Increase in power in late 21\textsuperscript{st} century agrees with previous results.
    \item In CESM1, increase in ENSO intensity is mainly strengthening of longer-period cycle.
    \item In CESM2, longer-period ENSO weakens after 2025.
    \end{itemize}
    \myfig{wavelet3.pdf}{Wavelet power spectrum for the Niño 3.4 index in the fully-forced CESM1 and CESM2 ensembles}{0.8}{wavelet2}
  \end{block}

  \vfill

  \begin{block}{Discussion}
    \begin{itemize}
    \item Rising greenhouse gas levels increase Pacific Ocean stratification, strengthening ENSO cycle.
    \item Aerosol influence is nonlinear because aerosol levels are not purely increasing.
    \item Stronger ENSO may lead to greater temperature variability and extreme weather.
    \item CESM1 and CESM2 conflict in their prediction of the changes to ENSO's frequency.
    \end{itemize}
  \end{block}

  \vfill

  \begin{block}{Limitations and Applications}
    Limitations:
    \begin{itemize}
    \item Niño 3.4 index shown to be inaccurate for some models \citep{cai2018increased}.
    \item CESM may contain biases.
    \end{itemize}
    Application: to improve our ability to predict ENSO and help people prepare for increased likelihood of extreme weather.
  \end{block}

  \vfill

  \begin{block}{Acknowledgments}
    \begin{itemize}
    \item This material is based upon work supported by the National Center for Atmospheric Research, which is a major facility sponsored by the National Science Foundation under Cooperative Agreement No. 1852977.
    \item Thank you to my teacher, my family, and my mentor!
    \item Software used: R, ncdf4, zoo, dplyr, ggplot2, WaveletComp, reshape2, nco.
    \end{itemize}
  \end{block}

  \vfill

  \begin{block}{Role of Mentor and Student}
    \begin{columns}[t]
      \column{.4\textwidth}
      Student:
      \begin{itemize}
      \item Analyze raw data on computer
      \item Produce graphics for analysis and publication
      \item Write documentation
      \item Identify key features of results
      \end{itemize}
      \column{.4\textwidth}
      Mentor:
      \begin{itemize}
      \item Review student writing
      \item Interpret results in the context of climatology
      \item Conduct parallel analysis
      \item Provide raw data from facility
      \end{itemize}
    \end{columns}
  \end{block}

  \vfill

  \begin{block}{References}
    \bibliographystyle{apalike}
    \scriptsize{\bibliography{references.bib}}
  \end{block}

  \vfill

\end{frame}
\end{document}
