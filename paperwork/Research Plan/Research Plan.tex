\documentclass{article}
\usepackage{adjustbox}
\usepackage[round]{natbib}

\title{The Effect of Natural and Anthropogenic Forcing on ENSO Amplitude in the CESM Large Ensemble}
\author{Benjamin Goldman}

\begin{document}

\maketitle

\section{Rationale}

El Ni\~{n}o/Southern Oscillation (ENSO) is a cyclical change in the temperature of the equatorial Pacific Ocean. It is accompanied by disturbances to the Walker Circulation and reduced upwelling in the eastern Pacific \citep{bjerknes1969atmospheric}. ENSO has extreme effects on global climate, affecting the intensity of rainfall in the Americas, the location of Rossby Waves, the paths of storms, and much more \citep{liu2007atmospheric}. Because of these critical effects, determining how climate change will affect ENSO is crucial.

Scientists have begun to research the effects of natural and anthropogenic forcing on ENSO. For example, \citet{zhu2017reduced} showed that during the Last Glacial Maximum, lower CO$_{2}$ levels reduced ENSO variability. \citet{levine2017impact} suggested that the Atlantic Multidecadal Oscillation (AMO) may also play a role in determining ENSO amplitude and demonstrated the difficulty in predicting long-term changes to ENSO. Researchers have also shown the effect of natural forcing on ENSO variability. \citet{emile2007nino} suggested the role of ENSO in the effect of solar and orbital forcing on the Earth's climate, and \citet{liu2007atmospheric} determined that ozone changes dampen CO$_{2}$-forced reduction of ENSO amplitude. The conflicting relationship of ozone and CO$_{2}$ has also been shown for ENSO diversity, as \citet{stevenson2019forced} showed that the effects of ozone and greenhouse gasses on the development of Eastern and Central Pacific ENSO events tend to cancel out. \citet{zheng2018response} showed that ENSO variability is highly dependent on internal variability, and therefore a model with a large ensemble is necessary to measure a robust signal.

\section{Research Goals}

\begin{itemize}

	\item Determine the effect of a range of climate factors on Nin\~{o} amplitude.
	\item Examine mechanisms responsible for observed effect(s)
	\item The following factors will be considered:
	\begin{itemize}
		\item Greenhouse gas emissions
		\item Aerosol emissions
		\item Biomass burning
		\item Land use/cover
		\item Ozone
	\end{itemize}
	
\end{itemize}

\section{Research Methods}

\subsection{Procedures}

\begin{enumerate}

	\item Data source: Community Earth System Model (CESM LME) output, publicly available but compiled by mentor
	\begin{itemize}
		\item Full-forcing and single-forcing ensembles using recorded radiative forcing from 1920-2005, and RCP 8.5 projections from 2005-2100
		\item 2000-year control run with pre-industrial forcing
	\end{itemize}
	\item The Python programming language with the Scipy package will be used to perform all calculations.
	\item Running centered 20-year Ni\~{n}o 3.4 variance will be calculated for each ensemble member and the PI control.

\end{enumerate}


\subsection{Data Analysis}

\begin{enumerate}

	\item Mean and standard error will be calculated for each ensemble.
	\item Probability Distribution Function (PDF) will be estimated for the control run.
	\item Ni\~{n}o 3.4 index for single-forcing ensembles will be subtracted from full-forcing ensemble to isolate changes associated with individual factors.
	\item Mean and standard error for full-forcing and single-forcing cases will be compared to that of the control.
	\item The bootstrap method will be used to compare the single and full-forcing distributions.
	\item Control Ni\~{n}o 3.4 variance PDF's will be calculated for high and low AMO and AMOC index using CVDP (Climate Variability Diagnostics Package) provided by mentor.
	\item Correlation between Ni\~{n}o variance and equatorial Pacific ocean tempecature will be calculated for the full-forcing ensemble.
	\item A mixed-layer heat budget analysis may be calculated to examine changes to ocean structure.
	
\end{enumerate}

\subsection{Risk and Safety}

No risk/safety issues relevant as all methods are digital.


\subsection{Role of Mentor and Student}

\begin{adjustbox}{width=1.5\textwidth, center=\textwidth}
\begin{tabular}{| l | c | r |}

\hline
Student & Shared & Mentor\\
\hline

Perform all calculations & Discuss and interpret results and data & Review results and methods\\
Produce plots and documentation &       & Suggest changes and future calculations\\
 & & Provide/compile datasets\\

\hline

\end{tabular}
\end{adjustbox}

\bibliographystyle{apa}
\bibliography{refs.bib}

\end{document}
