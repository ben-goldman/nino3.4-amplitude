\message{ !name(papernew.tex)}\documentclass[11pt]{article}
\usepackage[utf8]{inputenc}
\usepackage[T1]{fontenc}
\usepackage{graphicx}
\usepackage{grffile}
\usepackage{longtable}
\usepackage{wrapfig}
\usepackage{rotating}
\usepackage[normalem]{ulem}
\usepackage{amsmath}
\usepackage{textcomp}
\usepackage{amssymb}
\usepackage{capt-of}
\usepackage{hyperref}
\usepackage{natbib}
\usepackage[letterpaper, margin=1in]{geometry}
\setlength{\parskip}{0.35em}
\author{Benjamin Goldman}
\date{\today}
\title{The Impact of Anthropogenic Forcing on ENSO Amplitude}
\hypersetup{
 pdfauthor={Benjamin Goldman},
 pdftitle={The Impact of Anthropogenic Forcing on ENSO Amplitude},
 pdfkeywords={},
 pdfsubject={},
 pdfcreator={Emacs 27.2 (Org mode 9.5)}, 
 pdflang={English}}
\begin{document}

\message{ !name(papernew.tex) !offset(6) }
\section{Introduction}
\label{sec:org72159d6}
El Niño is the main mode of interanual climate variability, originating from an interaction between the atmosphere and water movement and temperature in the Pacific ocean \citep{bjerknes1969atmospheric}. The reasons for studying ENSO are clear, as ENSO has drastically affected climate patterns worldwide, modulating rainfall and temperature in nearly every continent \citep{ropelewski1987global}. For example, the recent 2015-2016 El Niño event contributed to record-breaking high temperatures and droughts in South America \citep{jimenez2016record}. At the same time, long-term anthropogenic greenhouse gas emissions are causing global temperatures to increase through a greenhouse effect. The effect of greenhouse emissions and other factors on ENSO intensity remains unclear.

The two major ways in which the earth's climate varies are climate change and climate variability. Climate change is defined as climate changes caused by external factors, most notably greenhouse gasses, natural and artificial aerosol emissions, land use changes, and stratospheric ozone changes. Greenhouse gas emissions have a clear impact on the earth's climate, global warming. Climate change is usually long-term. In contrast, internal variability is defined as changes to the earth's climate originating from natural climatic processes, such as ENSO, Pacific Decadal Oscillation (PDO), Atlantic Multidecadal Oscillation (AMO), and others. Climate variability occurs on much shorter timescales, and is usually cyclical and driven by feedback loops.

Research on the effect of external forcing on ENSO remains inconclusive, as results from similar studies conflict. \citet{nowack2017role} predicted an overall increase in Niño 3.4 standard deviation under a combination of 4xCO\(_2\) and interactive ozone forcing using single-model simulations, showing that greenhouse gasses increase the frequency of extreme ENSO by favoring a more Niño-like in the tropical Pacific, while ozone dampens this effect. In contrast, a few studies have found that ENSO amplitude decreases under global warming in certain coupled models \citep{kohyama2018weakening}.

However, other studies have failed to find any statistically significant ENSO response to external forcing \citep{stevenson2012significant}. Analysis using NCAR's CESM Large Ensemble shows an ensemble size of at least 15 models is necessary to attribute changes to Niño variance to external forcing and reject the null hypothesis that internal variability is responsible for changes to ENSO \citep{zheng2018response}. Several modes of internal variability have been shown to modulate ENSO, including the AMO \citep{levine2017impact} and Tropical Pacific Decadal Variability (TPDV) \citep{zheng2018response}. An analysis of the Max Plank Institute's Grand Ensemble as well as NCAR's CESM Large Ensemble suggests that 80$\backslash$% of changes to ENSO amplitude can be attributed to internal variability, but given a large enough ensemble, significant changes in ENSO amplitude due to climate change can be detected \citep{maher2018enso}.

In this study, we show that NCAR's CESM predicts significant increase in ENSO amplitude in the 21st century, and that greenhouse gasses and aerosol emissions play important roles in causing this increase. As in previous studies, the role of internal variability in conjunction with forcing was examined. We hypothesized that increased stratification in the future plays a large role in causing this predicted increase.


\message{ !name(papernew.tex) !offset(71) }

\end{document}
