% Created 2021-08-19 Thu 20:28
% Intended LaTeX compiler: pdflatex
\documentclass[11pt]{article}
\usepackage[utf8]{inputenc}
\usepackage[T1]{fontenc}
\usepackage{graphicx}
\usepackage{grffile}
\usepackage{longtable}
\usepackage{wrapfig}
\usepackage{rotating}
\usepackage[normalem]{ulem}
\usepackage{amsmath}
\usepackage{textcomp}
\usepackage{amssymb}
\usepackage{capt-of}
\usepackage{hyperref}
\author{Benjamin Goldman}
\date{\today}
\title{Notes for Paper}
\hypersetup{
 pdfauthor={Benjamin Goldman},
 pdftitle={Notes for Paper},
 pdfkeywords={},
 pdfsubject={},
 pdfcreator={Emacs 27.2 (Org mode 9.5)}, 
 pdflang={English}}
\begin{document}

\maketitle

\section{Introduction}
\label{sec:orge21977b}
\subsection{Rationale}
\label{sec:org22bff9d}
\subsubsection{Greenhouse emissions are increasing global temperatures (breifly state)}
\label{sec:org2d93704}
\subsubsection{ENSO is a change in the temperature of the pacific ocean with effects around the world}
\label{sec:org5775d0e}
\subsubsection{If climate change affects ENSO, then we need to find out \underline{now}}
\label{sec:org56c62ca}
\subsection{Review of literature:}
\label{sec:orge71ba68}
\subsubsection{Climate change affects ENSO}
\label{sec:orgddb4469}
\begin{enumerate}
\item Comparison of amplitude between many models with CMIP5 is mostly inconclusive
\label{sec:org0ff3528}
\item CESM-LE shows increase in amplitude that is significant
\label{sec:org409cf8d}
\end{enumerate}
\subsubsection{Role of multiple natural and artificial factors}
\label{sec:org7c2bb2f}
\begin{enumerate}
\item Greenhouse gasses may be increasing ENSO amplitude
\label{sec:org2ca497c}
\item aerosol emissions dampen GHG increase in amplitude
\label{sec:org21ed145}
\item What about everything else?
\label{sec:org5e6776a}
\item Not using a large ensemble mostly
\label{sec:org4781853}
\end{enumerate}
\subsection{Gap}
\label{sec:orga9e7bba}
\subsubsection{not enough analysis on how ghg, aer, and others affect ENSO amplitude}
\label{sec:org9c142a4}
\begin{enumerate}
\item how the two factors interact
\label{sec:org2cfc500}
\item how their effects look compared to internal variability
\label{sec:orgf657ec5}
\item What physical mechanisms mediate their effect
\label{sec:org99afee8}
\end{enumerate}
\subsubsection{Use some interesting methods: wavelet analysis}
\label{sec:org8f83908}
\subsection{Variables:}
\label{sec:org9774955}
\subsubsection{Independent variable: external forcing}
\label{sec:org2e804f0}
\begin{enumerate}
\item input climate models with observed and predicted greenhouse gas, aerosol, biomass burning, land use/cover, etc
\label{sec:org72954cd}
\item GHG:
\label{sec:org5622f92}
\begin{enumerate}
\item Observed and predicted to increase due to industrial burning
\label{sec:org6a16208}
\item Lead to higher air (and water) temperature due to greenhouse effect
\label{sec:orgda4de99}
\end{enumerate}
\item AER:
\label{sec:orgf592de7}
\begin{enumerate}
\item Aerosols produced by industry (smoke) and nature (volcanoes)
\label{sec:org9b2d67f}
\item Usually reduce global temperatures by reflecting oncoming sunlight
\label{sec:orge9600f1}
\item Location of emission may change as industry moves from US to China
\label{sec:orgca63294}
\end{enumerate}
\item BMB:
\label{sec:org2a56291}
\begin{enumerate}
\item Increased frequency of forest and grassland fires probably due to global warming
\label{sec:org5a531ca}
\item Releases CO\textsubscript{2} (GHG) and smoke (AER)
\label{sec:orgfc5ba1a}
\end{enumerate}
\item LULC:
\label{sec:orgc698b75}
\begin{enumerate}
\item Deforestation and desertification
\label{sec:orgde04da2}
\item Greenery reduces land temperature
\label{sec:org231cd88}
\item Greenery reduces CO\textsubscript{2} levels
\label{sec:org6abc24d}
\item Modifies water cycle as leaves increase evaporation
\label{sec:org2ceaf22}
\end{enumerate}
\end{enumerate}
\section{Data}
\label{sec:orgf91b064}
\subsection{CESM1}
\label{sec:orgdbe1af7}
\subsubsection{Full forcing}
\label{sec:org36a19d8}
\begin{enumerate}
\item Observed emissions levels 1920-2005
\label{sec:orgb6bc9c1}
\item RCP8.5 (worst case) emissions levels 2005-2100
\label{sec:org33be126}
\end{enumerate}
\subsubsection{Single forcing}
\label{sec:org7605e03}
\begin{enumerate}
\item All but one forcing group
\label{sec:org5ca7629}
\begin{enumerate}
\item GHG
\label{sec:org0bfc156}
\item AER
\label{sec:org1fc732e}
\item BMB
\label{sec:org31cca4f}
\item LULC
\label{sec:orgad3afc0}
\end{enumerate}
\item Observed 1920-2005
\label{sec:orgfc27146}
\item Predicted emissions 2005-2080
\label{sec:orgcd0ade9}
\end{enumerate}
\subsubsection{Control}
\label{sec:org3521a66}
\begin{enumerate}
\item 1850 fixed emissions levels
\label{sec:org78d4232}
\end{enumerate}
\subsection{CESM2}
\label{sec:org903fc5c}
\subsubsection{Higher ensemble size}
\label{sec:orgfa93ce4}
\subsubsection{Longer record length}
\label{sec:org6ef82de}
\subsubsection{Newer code}
\label{sec:org1bfcb6e}
\subsubsection{Full forcing}
\label{sec:orgad33236}
\begin{enumerate}
\item SMBB vs CMIP6 discrepancy
\label{sec:org7c1e55b}
\item Observed forcing levels 1850-2005
\label{sec:org82d9b1c}
\item Worst case emissions 2006-2100
\label{sec:org849b535}
\end{enumerate}
\subsubsection{Single Forcing}
\label{sec:org373f00c}
\begin{enumerate}
\item Similar to CESM1
\label{sec:org690c5d6}
\item Not all members are done yet
\label{sec:org8ea3d2f}
\end{enumerate}
\subsubsection{Control}
\label{sec:orgb301cb6}
\begin{enumerate}
\item Similar to CESM1
\label{sec:org735b3e3}
\end{enumerate}
\subsection{Observed data}
\label{sec:orgeb1c1df}
\subsubsection{Might mention in introduction?}
\label{sec:orga6c7ed4}
\subsubsection{Used for introductory presentation figures, not yet for actual methods}
\label{sec:org9778c23}
\section{Methods and Results}
\label{sec:org1eb5e61}
\subsection{For all ensembles and all variables:}
\label{sec:orgc5525b4}
\subsubsection{Subtract ensemble mean from each member to derive internal variability i.e. ENSO}
\label{sec:orgba6e56a}
\subsection{Niño 3.4 index}
\label{sec:orgf22a460}
\subsubsection{Mean temperature in Niño 3.4 region}
\label{sec:org7245a90}
\subsubsection{20-year rolling variance for each member}
\label{sec:org084d966}
\subsubsection{Ensemble statistics: mean and standard error (dependent on ensemble size)}
\label{sec:orgdf5efb6}
\subsubsection{Control?}
\label{sec:orgb67f98a}
\subsubsection{Figure 1: CESM1 and CESM2 FF Niño 3.4 variance ensemble means}
\label{sec:orgdb13ad6}
\subsubsection{Explanation:}
\label{sec:org1c938fa}
\begin{enumerate}
\item CESM1
\label{sec:org20d12a8}
\begin{enumerate}
\item Starts with low variance, steeply increasing from 1950-2050, dropping slightly after 2050
\label{sec:orga9e35ea}
\item Probably nonlinear response to forcing: does not keep increasing
\label{sec:org97cb690}
\begin{enumerate}
\item Are emissions slowing down post-2050?
\label{sec:org42ff186}
\end{enumerate}
\end{enumerate}
\item CESM2
\label{sec:org5378e06}
\begin{enumerate}
\item Exponential? increase in variance until 2025, then sharp decrease
\label{sec:org3dea794}
\begin{enumerate}
\item Why?
\label{sec:org487e458}
\end{enumerate}
\end{enumerate}
\end{enumerate}
\subsection{Single forcing ensembles}
\label{sec:org85701d9}
\subsubsection{SF ensembles for CESM1 and CESM2}
\label{sec:org863b2ac}
\subsubsection{All-but one}
\label{sec:org55682bb}
\subsubsection{Figure 2: ENSO amplitude index for SF and FF models}
\label{sec:org9520816}
\subsubsection{Explanation}
\label{sec:orge8bf1b3}
\begin{enumerate}
\item -BMB and -LULC means are pretty close to FF mean, signaling little impact
\label{sec:orge62b262}
\item -AER and -GHG means are mostly unchanged, signaling stronger impact
\label{sec:org778cb4f}
\end{enumerate}
\subsubsection{Bootstrap}
\label{sec:org4a237e8}
\begin{enumerate}
\item Subtract random SF member from random FF member
\label{sec:org54d0446}
\item Leaves behind isolated effect of single factor
\label{sec:orgf1b88f4}
\item aeu
\label{sec:orgbab39ac}
\end{enumerate}
\end{document}
