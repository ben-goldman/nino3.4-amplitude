% Created 2021-05-10 Mon 19:08
% Intended LaTeX compiler: pdflatex
\documentclass[11pt]{article}
\usepackage[utf8]{inputenc}
\usepackage[T1]{fontenc}
\usepackage{graphicx}
\usepackage{grffile}
\usepackage{longtable}
\usepackage{wrapfig}
\usepackage{rotating}
\usepackage[normalem]{ulem}
\usepackage{amsmath}
\usepackage{textcomp}
\usepackage{amssymb}
\usepackage{capt-of}
\usepackage{hyperref}
\date{\today}
\title{Abstract}
\hypersetup{
 pdfauthor={John Doe},
 pdftitle={Abstract},
 pdfkeywords={},
 pdfsubject={},
 pdfcreator={Emacs 27.2 (Org mode 9.5)}, 
 pdflang={English}}
\begin{document}

\begin{center}

{\bf The Impact of Industrial Emissions on ENSO Amplitude}
{\bf Benjamin Goldman}

\end{center}

The El Niño/Southern Oscillation (ENSO) is the dominant mode of interannual climate variability, with substantial associated global socio-economic impacts. Due to their significance, shifts in ENSO under climate change also have the potential to substantial impact human society and natural ecosystems. However, it is currently unclear what effect greenhouse gas (GHG) and industrial aerosol (AER) emissions will have on ENSO and even less clear what effect a combination of these factors might have when changing in tandem. This study examines transient changes to ENSO variance under a variety of forcing scenarios using the CESM1 Large and Single-Forcing Ensembles. These multi-member ensembles span the historical record (1920-2005) and much of the 21st C (2006-2080 for GHG/AER). A 2000-year pre-industrial (PI) control simulation is used to account for model drift and 20-year running variance of the Niño 3.4 SST index is used as a proxy for ENSO variance. The ensemble mean and standard error of each ensemble is calculated, to estimate the statistical significance of simulated changes. We identify significant increases in variance under full-forcing conditions during the historical record and attribute these mainly to changes in GHG, with the potential emergence of AER-driven increases in the decades to come. Computing detrended and smoothed correlation between 20-year variance of the Niño 3.4 index and ocean temperature in the tropical Pacific Ocean reveals that stratification caused by global warming is connected with increased ENSO amplitude.
\end{document}
