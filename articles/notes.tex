% Created 2021-07-08 Thu 18:46
% Intended LaTeX compiler: pdflatex
\documentclass[11pt]{article}
\usepackage[utf8]{inputenc}
\usepackage[T1]{fontenc}
\usepackage{graphicx}
\usepackage{grffile}
\usepackage{longtable}
\usepackage{wrapfig}
\usepackage{rotating}
\usepackage[normalem]{ulem}
\usepackage{amsmath}
\usepackage{textcomp}
\usepackage{amssymb}
\usepackage{capt-of}
\usepackage{hyperref}
\usepackage{enumitem}
\setlist{noitemsep}
\author{Benjamin Goldman}
\date{\today}
\title{Notes}
\hypersetup{
 pdfauthor={Benjamin Goldman},
 pdftitle={Notes},
 pdfkeywords={},
 pdfsubject={},
 pdfcreator={Emacs 27.2 (Org mode 9.5)}, 
 pdflang={English}}
\begin{document}

\maketitle
\tableofcontents



\section{\cite{bjerknes1969atmospheric}}
\label{sec:org2ba3d44}

\begin{itemize}
\item First big paper on ENSO having a big impact
\item connected changes in ocean currents to Walker Circulation
\item ENSO phase affects behavior of the Indian Ocean monsoon.
\end{itemize}

\section{\cite{an2017feedback}}
\label{sec:org301fc0b}

\begin{itemize}
\item Used SVD (Singular Value Decomposition) together with the Mixed Layer Heat Budget Analysis to look at which feedbacks contributed most to ENSO's variation between models
\item Influence of thermocline feedback is determined by how strongly equatorial horizontal winds affect the slope of the thermocline.
\end{itemize}

\section{{\bfseries\sffamily TODO} \cite{boer2000transient}}
\label{sec:org0861beb}

\section{\cite{cai2018increased}}
\label{sec:org4e917df}

\begin{itemize}
\item Increased ENSO variance in most CMIP5 models in EP ENSO center.
\item Likely caused by greenhouse gases
\item Higher ocean stratification allows for stronger communication between atmospheric and oceanic temperatures.
\item Used EOF analysis.
\end{itemize}

\section{{\bfseries\sffamily TODO} \cite{chen2015causes}}
\label{sec:org1579516}

\section{\cite{chen2017possible}}
\label{sec:org0513968}

\begin{itemize}
\item Models are disagreeing on ENSO in the future because they have different representations of the mechanics and mean state of the Pacific subtropical cell
\end{itemize}

\section{{\bfseries\sffamily TODO} \cite{deser2020isolating}}
\label{sec:org1861fc1}

\begin{itemize}
\item Main documentation for CESM1 Single Forcing Ensemble
\end{itemize}

\section{{\bfseries\sffamily TODO} \cite{dewitte2012reinterpreting}}
\label{sec:org0dc6c59}

\section{\cite{emile2007nino}}
\label{sec:org66963df}

\begin{itemize}
\item Analyzed wavelet power spectrum of ENSO variability in models forced by sunspot and orbital changes
\item Orbital changes increase long-term ENSO variability
\item It is possible that ENSO was the mechanic that allowed prehistoric solar/orbital changes to control the earth's climate
\end{itemize}

\section{\cite{graham2014effectiveness}}
\label{sec:orgbe0857b}

\begin{itemize}
\item tested how accurate the Bjerknes Stability Index is at measuring the mechanics of ENSO in a couple models
\item BJ index overestimates the importance of the Thermocline feedback.
\item BJ index assumes that terms should be linear when combined, but they actually aren't.
\end{itemize}

\section{{\bfseries\sffamily TODO} \cite{torrence1998practical}}
\label{sec:org96cb080}

\begin{itemize}
\item How to use wavelets to estimate power spectrum in timeseries.
\item Uses ENSO data \emph{very niiceee}
\item Windowed Fourier Transform sucks butt because it is dependent on a time step parameter that can muck with the results depending on which value you choose.
\item A wavelet is a short \textbf{\emph{blirp}} of a wave with a mean of zero and finite amplitude/frequency and limited time domain.
\end{itemize}



\bibliographystyle{apalike}
\bibliography{references}
\end{document}
